%nag warns a lot about tikzposter.
%\RequirePackage[l2tabu, orthodox]{nag}
\documentclass[blockverticalspace=3cm]{tikzposter}
\input{preamble/packages}
\input{preamble/redac}
\input{preamble/math_basics}
%Decision Theory (MCDA and SC)
\newcommand{\allalts}{\mathscr{A}}
\newcommand{\allcrits}{\mathscr{C}}
\newcommand{\alts}{A}
\newcommand{\dm}{\textcolor{violet}{\ensuremath{i}}}
\newcommand{\model}{\textcolor{cyan}{\eta}}
\newcommand{\allF}{\mathscr{F}}
\newcommand{\allvoters}{\mathscr{N}}
\newcommand{\voters}{N}
\newcommand{\allprofs}{\boldsymbol{\mathcal{R}}}
\newcommand{\prof}{\boldsymbol{R}}
\newcommand{\linors}{\mathscr{L}(\allalts)}
%Thanks to https://tex.stackexchange.com/q/154549
	%\makeatletter
	%\def\@myRgood@#1#2{\mathrel{R^X_{#2}}}
	%\def\myRgood{\@ifnextchar_{\@myRgood@}{\mathrel{R^X}}}
	%\makeatother

%Deliberated Judgment
\newcommand{\props}{\textcolor{red}{T}}
\newcommand{\prop}{\textcolor{red}{t}}
\newcommand{\allargs}{S^*}
\newcommand{\args}{\textcolor{blue}{S}}
\newcommand{\ar}[1][]{%
	\textcolor{blue}{%
		\ifx\\#1\\%
			s
		\else
			s_#1
		\fi%
	}%
}
\newcommand{\ileadsto}{\textcolor{violet}{⇝}}
\newcommand{\ibeatse}{\textcolor{violet}{⊳_\exists}}
\newcommand{\nibeatse}{\textcolor{violet}{⋫_\exists}}
\newcommand{\ibeatsst}{⊳_\forall}
\newcommand{\nibeatsst}{⋫_\forall}
\newcommand{\mleadsto}[1][\eta]{⇝_{#1}}
\newcommand{\mbeats}[1][\eta]{\textcolor{cyan}{⊳_{#1}}}
\newcommand{\ibeatseinv}{⊳_\exists^{-1}}

%Logic
\newcommand{\ltru}{\texttt{T}}
\newcommand{\lfal}{\texttt{F}}


\input{preamble/draw}

\title{Studying Deliberated Judgments}
\institute{LAMSADE, Université Paris-Dauphine}
\author{Olivier Cailloux}

\begin{document}
\maketitle[titletotopverticalspace=10cm]

\begin{columns}
	\column{0.4}
		\block{Context and goal of this poster}{
			\begin{tikzpicture}[remember picture,overlay]
				\path (current page.north west) ++(1.5cm, -1cm) node[anchor=north west, inner sep=0] (first) {
					\includegraphics[height=7cm]{LAMSADE95.jpg}
				};
%				\path (current page.north) ++(0cm, -1cm) node[anchor=north, inner sep=0] (second) {
%					\includegraphics[height=7cm]{dauphine_psl2018.png}
%				};
%				\path (current page.north east) ++(-1.5cm, -1cm) node[anchor=north east, inner sep=0] (third) {
%					\includegraphics[height=7cm]{LOGO-PSL-nov-2017.pdf}
%				};
				\path (current page.north east) ++(-1.5cm, -1cm) node[anchor=north east, inner sep=0] {
					\includegraphics[height=7cm]{dauphine_psl2018.png}
				};
			\end{tikzpicture}%
		%
			\begin{tikzpicture}[remember picture,overlay]
				\path (current page.south west) ++(1.5cm, 0.6cm) node[anchor=south west, text width=33cm] {
					Olivier Cailloux and Yves Meinard. \emph{A formal framework for deliberated judgment}. To appear in Theory and Decision.\\ \url{https://github.com/oliviercailloux/formal-framework-dj}.
				};
			\end{tikzpicture}%
		%
			\textbf{Context}
			\begin{itemize}
				\item Deliberation facing a decision problem
				\item Considering an individual \dm
			\end{itemize}

			\textbf{Goal}
			\begin{itemize}
				\item Introduce the notion of Deliberated Judgment
				\item Motivate studying it
				\item Sketch how
			\end{itemize}
		}
	\column{0.6}
		\block{Deliberated judgment: a missing conception of “preference”}{
			\begin{itemize}
				\item Descriptive approach
				\begin{itemize}
					\item Observe people’s epistemic position / choice without interference
				\end{itemize}
				\item Normative approach
				\begin{itemize}
					\item How you ought to reason / choose
					\item Can’t be validated through observation of individuals
				\end{itemize}
				\item \emph{Deliberated} judgment (or preference)
				\begin{itemize}
					\item \dm’s position after having considered all arguments
				\end{itemize}
			\end{itemize}
		}
\end{columns}

\begin{columns}
	\column{0.65}
	\block{Predicting deliberation issue}{
		\newlength{\seplott}
		\setlength{\seplott}{3em}
		\begin{itemize}
			\item Choose between L1 and L2 (Kahneman and Tversky, The Psychology of
Preferences, Scientific American, January 1982, 246, 160-73.)\\
			\begin{tikzpicture}[grow'=right, sibling distance=3cm, level 1/.style ={level distance=4cm}, level 2/.style ={level distance=8cm}, remember picture]
				\path node (l1) {L1} child {
					node {200 €} child {
						node {50 €} edge from parent node[above] {100\%}
					}
				};
				\path (l1-1-1.east) ++ (\seplott, 0) node (ql12) {VS};
				\path (ql12) ++ (\seplott, 0) node[anchor=west] (l2) {L2} child {
					node {200 €} child {
						node {0 €} edge from parent node[above] {75\%}
					} child {
						node {200 €} edge from parent node[below] {25\%}
					}
				};
			\end{tikzpicture}
			\item Choose between L3 and L4\\
			\begin{tikzpicture}[grow'=right, sibling distance=3cm, level 1/.style ={level distance=4cm}, level 2/.style ={level distance=8cm}, remember picture]
				\path node (l3) {L3} child {
					node {400 €} child {
						node {−150 €} edge from parent node[above] {100\%}
					}
				};
				\path[overlay] (l3-1-1 -| ql12) node (ql34) {VS};
				\path (ql34) ++ (\seplott, 0) node[anchor=west] (l4) {L4} child {
					node {400 €} child {
						node {−0 €} edge from parent node[above] {25\%}
					} child {
						node {−200 €} edge from parent node[below] {75\%}
					}
				};
			\end{tikzpicture}
		\end{itemize}
	}
	\column{0.35}
	\block{Disc}{
		\begin{itemize}
			\item First observation (Bernouilli): don’t be content with maximizing (untransformed) expected revenue!
			\item Second observation: \dm{} could be intuitively attracted by L1 $\succ$ L2 and L3 $\succ$ L4 (Allais’s problem)
			\item Including Savage
			\item And might change her mind when given a reasoning pro expected utility
			\item “There is, of course, an important sense in which preferences, being entirely subjective, cannot be in error”
			\item … “but in a different, more subtle sense they can be.” (Savage, \emph{The Foundations of Statistics})
			\item[⇒]Systematic decision principles might help deliberate
		\end{itemize}
	}
\end{columns}

\begin{columns}
	\column{0.45}
		\block{Study deliberated judgment}{
			The proposed research program aims at the following.
			\begin{enumerate}
				\item Define \ac{DJ} of \dm{} formally
				\begin{itemize}
					\item Given a set of arguments
					\item[⇒] The position that is stable facing counter-arguments
				\end{itemize}
				\item Define the concept of a model of \dm’s \ac{DJ}
				\begin{itemize}
					\item[⇒] A model articulates claims concerning \dm’s \ac{DJ} and argues for its claim
				\end{itemize}
				\item Define validity of a model
				\begin{itemize}
					\item[⇒] Correctly captures \dm’s \ac{DJ}
				\end{itemize}
				\item Study conditions for falsifying models using observable data only
				\begin{itemize}
					\item[⇒] Let models debate, use \dm{} as a judge
				\end{itemize}
			\end{enumerate}
			\vspace{0.5em}
			We obtain a theorem of the following form.\par
			\emph{If the decision situation $(\props, \args, \ileadsto, \ibeatse, \nibeatse)$ satisfies conditions 1 to 4: an operationally valid model exists; and any operationally valid model is valid.}
		}
	\column{0.55}
		\block{Example of a situation and a model of it}{
			\begin{tabularx}{40cm}{LYl}
				\toprule
				\text{Notation}	&\text{Here}	&Description\\
				\midrule
				\props	&\set{\prop}	&The topic, containing propositions about which \dm deliberates\\
				\args	&\set{\ar, \ar[1], \ar[2], \ar[3]}	&The arguments\\
				{\ileadsto} \subseteq \args × \props	&\set{(\ar, \prop), (\ar[1], \prop)}	&Support as considered by \dm\\
				{\ibeatse} \subseteq \args × \args\hspace{1cm}\mbox{}	&\set{(\ar[2], \ar[1])}	&$\ar[2] \mathrel{\ibeatse} \ar[1]$ iff \dm{} sometimes considers that $\ar[2]$ trumps $\ar[1]$\\
				{\mbeats} \subseteq \args × \args	&\set{(\ar[3], \ar[2])}	&Trump situations as considered by the model $\model$\\
				\bottomrule
			\end{tabularx}
			\vspace{1em}
			\begin{center}
			\begin{tikzpicture}
				\path node (eq) {weather f.\ predicts so ($\ar[1]$)};
				\path (eq.east) node[anchor=west] (eql) {$\ileadsto$};
				\path (eql.east) node[anchor=west] (eqr) {rain tomorrow ($\prop$)};
				\path (eq.south) ++(0, -\BDNodeSep) node[anchor=north] (wrong) {weather forecast is often wrong ($\ar[2]$)};
				\path (wrong) edge[/Beliefs/attack] node[right] {$\ibeatse, \nibeatse$} (eq);
				\path (wrong.south) ++(0, -\BDNodeSep) node[anchor=north] (right) {weather forecast is more often right ($\ar[3]$)};
				\path (right) edge[/Beliefs/attack] node[right] {$\ibeatse, \mbeats$} (wrong);
				
				\path (eqr.east) node[anchor=west] (complexl) {\reflectbox{$\ileadsto$}};
				\path (complexl.base east) node[anchor=base west] (complex) {complex arg. ($\ar$)};
			\end{tikzpicture}
			\end{center}
%		\raggedleft{Example of a }
		}
\end{columns}

\begin{columns}
	\column{0.5}
		\block{Application: test axioms of decision theory}{
			\begin{itemize}
				\item Axioms considered appropriate normatively?
				\begin{itemize}
					\item But some (Allais, Ellsberg) disagree
				\end{itemize}
				\item Proposal: build models resting on those axioms
				\item Test models: their convincing power will give us indications about the reasonableness of the axioms for “normal” people (meaning, not scientists studying decision theory)
			\end{itemize}
		}
	\column{0.5}
		\block{Application: test conceptions of justice}{
			\begin{itemize}
				\item Philosophers have proposed sophisticated conceptions of justice (Rawls, Nozick, …)
				\item Individual’s shallow intuitions about justice are observed and used to confront Rawls or others (Experimental Social Choice)
				\item Proposal: study reactions of individuals to arguments of philosophers rather than just shallow intuitions
				\item Move towards Reflective equilibrium (Goodman, Rawls)
			\end{itemize}
		}
\end{columns}
\end{document}

